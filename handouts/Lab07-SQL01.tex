\documentclass[12pt]{scrartcl}

\input{../preamble.tex}

\title{CSCE 156 -- Computer Science II}
\subtitle{Lab 07 - SQL I}
\author{Dr.\ Chris Bourke}
\date{~}

\begin{document}

\maketitle

\section*{Prior to Lab}

\begin{enumerate}
  \item Review this laboratory handout prior to lab.
  \item Create a MySQL account by logging in here: \url{https://cse-apps.unl.edu/amu/amu/login}
  \item Change the account password using the following directions at: 
  	\url{http://cse.unl.edu/faq-section/unix-linux#node-302}
  \item Review the supplemental SQL Cheat Sheet for the Album Database
  \item This is a long lab but it can be completed if you prepare 
	properly.  Review the following materials:
  \begin{itemize}
    \item Information About Databases and Tables \\
    \url{http://dev.mysql.com/doc/refman/5.6/en/getting-information.html}
    \item Connecting to MySQL from the command line: \\ 
    \url{http://dev.mysql.com/doc/refman/5.6/en/connecting-disconnecting.html}
    \item Retrieving data\\ \url{http://www.w3schools.com/sql/sql_select.asp}
    \item Conditional clause	\\ \url{http://www.w3schools.com/sql/sql_where.asp} 
    \item Inserting data	\\ \url{http://www.w3schools.com/sql/sql_insert.asp} 
    \item Deleting data	\\ 
    \url{http://www.w3schools.com/sql/sql_delete.asp} 
    \item Updating data	\\ 
    \url{http://www.w3schools.com/sql/sql_update.asp} 
    \item \mintinline{sql}{count()}	\\ 
    \url{http://www.w3schools.com/sql/sql_func_count.asp}
    \item \mintinline{sql}{max()}	\\ \url{http://www.w3schools.com/sql/sql_func_max.asp}
    \item \mintinline{sql}{min()}	\\ \url{http://www.w3schools.com/sql/sql_func_min.asp}
    \item Joining tables	inner join	\\ 
    \url{http://www.w3schools.com/sql/sql_join_inner.asp} 
	\item left join	\\ 
    \url{http://www.w3schools.com/sql/sql_join_left.asp} 
	\item right join	\\ 
    \url{http://www.w3schools.com/sql/sql_join_right.asp} 
  \end{itemize}  
\end{enumerate}

\section*{Lab Objectives \& Topics}
Following the lab, you should be able to:
\begin{itemize}
  \item Connect to a database and execute queries
  \item Perform basic Create, retrieve, update, and delete (CRUD) operations
  \item Understand more complex queries using Joins and Aggregate functions
\end{itemize}

\input{../peerText.tex}

\section*{Getting Started}

Material for this lab is available for download on the course website.
%Clone the project code for this lab from GitHub in Eclipse using the
%URL, \url{https://github.com/cbourke/CSCE156-Lab06-Polymorphism}.
%Refer to Lab 01 for instructions on how to clone a project from GitHub.

\section*{Querying a Database}

You will be connecting to a remote MySQL database server on CSE 
and executing several queries.  The queries you will be performing 
involve a database that contains data about various music albums, songs
and the artists involved.  The database structure is illustrated 
in the ER (Entity-Relation) diagram in Figure \ref{figure:albumDB}.

\begin{figure}[h]
\centering
\includegraphics[scale=.650]{sql/albums}
\caption{Albums Database}
\label{figure:albumDB}
\end{figure}

\subsection*{Importing the Database}

You will need to ``install'' the Albums database and data into your 
own database on CSE.  Note: database in this sense is just a collection 
of related tables; on CSE you only have access to one actual 
database--the database named after your CSE login.  To import the 
Albums database, you can either a) simply run the \mintinline{text}{albums.sql}
script (you may need to add a \mintinline{sql}{use cselogin;} first)
in MySQL Workbench; or b) from the command line:

\begin{enumerate}
  \item Make sure the \mintinline{text}{albums.sql} DDL file (Data 
  	Description Language) is on your Z drive.  
  \item From the command line (via PuTTY), execute the following:
  
  \mintinline{text}{mysql -u username -p username < albums.sql}
  
  where \mintinline{text}{username} is replaced with your CSE login.
  Enter your MySQL password.  This redirects the contents of the 
  \mintinline{text}{albums.sql} file (a collection of SQL commands) 
  to the mysql command line interface, creating all the tables and 
  inserting all the data necessary.
\end{enumerate}
  
\subsection*{Executing Queries}

You may use any interface to your MySQL database that you wish, but 
we recommend that you use MySQL Workbench.  You can download and
install it from here: \url{https://www.mysql.com/products/workbench/}.
Otherwise, it is available on the CSE lab computers.

\begin{enumerate}
  \item Launch MySQL Workbench
  \item From the quick launch menu select ``Open Connection to Start Querying''
  \item Enter the host name (cse.unl.edu), username (your cse login) 
    and enter your sql password; click ``OK''
  \item You can now enter queries and execute them (follow the menu options)
\end{enumerate}

Execute the queries in the worksheet and demonstrate them to a
lab instructor.  Instead of writing the answers by hand, you may
simply type them in the worksheet provided.
  
\subsection*{SQL Supplemental Cheat Sheet}

For your benefit, we have created a supplemental SQL cheat sheet that 
you may reference.  It contains many of the major types of queries 
along with a practical application using the Album database.

\end{document}
